\input{formats/diploma.tex}

\usepackage[utf8]{inputenc}                % Кодировка
\usepackage[T1,T2A]{fontenc}
\usepackage[main=russian, english]{babel}  % Русский язык
\usepackage[pdftex]{graphicx}              % Картинки
\usepackage{indentfirst}                   % Отступ перед абзацами

\usepackage[unicode]{hyperref}                                         % Ссылки и русские закладки
\hypersetup {                                                          %
    pdftitle={Курсовая работа},                                        % Название документа
    pdfsubject={Параллельное что-то},                                  % Тема документа
    pdfauthor={Егоров Кирилл Юлианович},                               % Автор документа
    pdfcreator={Кафедра системного анализа ВМК МГУ},                   % Создатель документа
    pdfproducer={LaTeX},                                               % Программа, создавшая документ
    hidelinks                                                          % Скрывает рамку вокруг ссылок
}

\begin{document}
    \thispagestyle{empty}
\begin{center}
    \ \vspace{-3cm}

    \includegraphics[width=0.5\textwidth]{title_page/msu.eps}\\

    {\small{\scshape  Московский государственный университет имени М.~В.~Ломоносова}\\
    Факультет вычислительной математики и кибернетики\\
    Кафедра системного анализа}

    \vfill

    {\Large Егоров Кирилл Юлианович}

    \vspace{1cm}

    {\LARGE\bfseries Построение параллельных алгоритмов\parдля решения задачи кратчайшего пути}

    \vspace{1.5cm}

    {\scshape Курсовая работа}
\end{center}

\vspace{3cm}

\begin{flushright}
    \large
    \textit{Научный руководитель}\\
    к.ф.-м.н., доцент И.~В.~Востриков
\end{flushright}

\vfill

\begin{center}
    Москва, 2022
\end{center}

\clearpage
    \tableofcontents
    \clearpage

    \section{Введение}
    \section{Постановка задачи}
    \section{Классические решения}
        \subsection{Алгоритм Дейкстры}
        \subsection{Алгоритм Беллмана--Форда}
    \section{Параллельные алгоритмы}
        \subsection{Параллелелизм на ядрах процессора}
        \subsection{Параллелелизм на многонодной установке}
        \subsection{Параллелелизм на графической видеокарте}
    \section{Описание общего программного решения}
    \section{Заключение}
    
    \begin{thebibliography}{9}
        \bibitem{} Bellman,R. DYNAMIC PROGRAMMING. Princeton University Press, New Jersey, 1957.
        \bibitem{} Kirk, D. E. OPTIMAL CONTROL THEORY: AN INTRODUCTION. Prentice Hall, 1970.
        \bibitem{} Ross, I. M. PONTRYAGIN'S PRINCIPLE, in Ch.2 of A Primer on Pontryagin's Principle in Optimal Control, Collegiate Publishers, San Francisco, 2015.
    \end{thebibliography}
\end{document}


