Работа посвящена разработке параллельных алгоритмов для поиска оптимального по быстродействию пути между двумя точками на некоторой ограниченной территории с препядствиями, которые мешают движению.
В~работе эти препядствия представляют собой произвольные области, в которых невозможно дальнейшее движение.

Для представления в виде классической \textit{задачи кратчайшего пути} исходная задача дискретизуются.
Численное решение уравнения Гамильтона--Якоби--Беллмана, к которому сводится непрерывный вариант возможно
 (по крайней мере, в \cite{gyb} предложен способ решения после нескольких приближений и дискретизации уравнения), но дискретизация задачи в целом~--- более популярное решение. 

Необходимость решения задачи кратчайшего пути в двумерном пространстве возникает в различных сферах, но в первую очередь в транспортной и логистической.
Сложно представить будущее, где проблема быстрой доставки товаров или безопасного перемещения людей перестала быть актуальной.
Данная задача является классической и традиционно решается методом динамического программирования\cite{bellman}.

Последовательные алгоритмы, решающие данную задачу хорошо изучены и представлены в статьях \cite{dijkstra} и \cite{review}.
Эти алгоритмы были адаптированы под поставленную задачу, и на их основе были разработаны параллельные варианты.

Было уделено вниманию различным способам организации параллелелизма: на многоядерном процессоре, на графическом процессоре, на многонодной компьютерной установке.
Для каждого способа описан соответствующий алгоритм, для двух видов параллелелизма разработаны программы и замерены времена работы.