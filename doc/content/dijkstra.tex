Алгоритму Дейкстры~--- это самый популярный алгоритм поиска кратчайшего пути. Из-за меньшей, чем у алгоритма Беллмана--Форда алгоритмической сложности, данный алгоритм позволяет производить вычисления на сетках на порядок большего размера. 

Алгоритм использует две дополнительные структуры: множество обработанных узлов и можество граничных узлов. Данные структуры имплементированы как хэш-таблицы.

Ниже приведено описание алгоритма:
\begin{enumerate}
    \item На начало алгоритма в множестве граничных узлов находится только целевой узел. 
    \item На каждой итерации алгоритма из множества граничных узлов выбирается узел с минимальной маркировкой.
    \item Этот узел добавляется в множество обработанных узлов и удаляется из граничного множества.
    \item Затем из возможного множества выбираются узлы, которые не находятся в множестве обработанных узлов. Эти узлы добавляются в множество граничных вершин, маркировка таких вершин обновляется.
\end{enumerate}
Алгоритм останавливается в случае, если на некоторой итерации алгоритма был выбран начальный узел $(i_0,j_0)$.

\subsection{Распараллелевание на CPU}

Основной затратной операцией в приведенном алгоритме является поиск минимума на каждой итерации.
Поэтому параллелелизм будет встроен именно в этот этап.

Для этого предлагается хранить граничное множество в $C$ хэш-таблицах.
Добавлять узел в наименее наполненную таблицу.
Тогда каждый процесс ищет минимум в своей хэш-таблице, а затем основной процесс ищет минимум в массиве из $C$ элементов.

\subsection{Распараллелевание на многонодной установке}

Для обеспечения синхронизации снова пользуемся инкрементальным обновлением.
Заметим, что за одну итерацию алгоритма происходит одно добавление в множество обработанных вершин, и не более 7 добавлений в граничное множество.
Этой информации достаточно для поддержания всех вычислителей в актуальном состоянии, поэтому она должна сообщаться сервисам-вычислителям после каждой проведенной итерации.