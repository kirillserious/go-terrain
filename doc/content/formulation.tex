Рассмотрим управляемый объект, положение которого задается динамической системой дифференциальных уравнений
\begin{equation}\label{first}
    \frac{dx}{dt} = f(t, x, u)
\end{equation}
на промежутке $t_0 \leqslant t \leqslant t_1$ с заданным начальным состоянием $x(t_0) = x^0$.

Такое описание естесственно для движений, подчиняющихся второму закону Ньютона.
В нашем случае $x = [x_1, x_2, v_1, v_2]\in \mathbb{R}^4,\;v_i=\dot x_i,\,i=\{1,2\}$.
Наложим на фазовые координаты следующие ограничения:
\[
    [x_1, x_2] \in \Omega, \quad [v_1, v_2] \in \Omega_v.
\]

Для задачи Коши (\ref{first}) поставим задачу поиска управления $u \in U$, минимизирующего следующий интегральный функционал:
\begin{equation}
    J(u) = \int\limits_{t_0}^{t_1} g(x(t), u(t))\,dt \rightarrow \min\limits_{u\in U}.
\end{equation}

Будем пользоваться методом динамического программирования. Введем функцию цены
\[
    \begin{aligned}
        &V(t,x) = \min\limits_{u \in U}\int\limits_{t}^{t_1}g(x(t), u(t))\,dt,\\
        &V(t_1,x^1) = 0.
    \end{aligned}
\]
Согласно \cite{optcontrol}, если предполагать непрерывную дифференцируемость функции цены, решение задачи равносильно решению уравнения Гамильтона--Якоби--Беллмана:
\begin{equation}\label{gyb}
    \min\limits_{u(t) \in U}\left\{
        g(x,u) + \sum\limits_{i=1}^{n}\frac{\partial V(x)}{\partial x_i} f_i(x,u)
    \right\} = 0.
\end{equation}

\textit{Замечание.}
Мы рассматриваем частный случай этой задачи --- задачу быстродейсвия, то есть задачу с функционалом качества $g(x,u) \equiv 1$. В этих условиях уравнение Гамильтона--Якоби--Беллмана (\ref{gyb}) будет записано в следующем виде:
\[
    \min\limits_{u\in U}\langle \nabla t(x), f(x,u) \rangle = -1.
\]

Вместо дискретизации и последующего решения получившегося уравнения мы дискретизируем исходную задачу для приведения ее к виду задачи кратчайшего пути.

Пусть $\Pi$~--- минимальный прямоугольник, вмещающий в себя целиком множество $\Omega$. Введем на нем равномерную сетку с шагом $\varepsilon$:
\begin{equation}
    (i,j):\;1\leqslant i \leqslant N,\,1\leqslant i \leqslant M
\end{equation}
\begin{equation}
    \Xi = \left\{ (x_i , y_j) \in \Pi , x_i = \varepsilon\frac{i}{N} , y_j = \varepsilon\frac{j}{M} \right\}.
\end{equation}

Мы будем считать, что $x^0,x^1 \in \Xi$. Перемещение возможно только в ``соседние'' узлы сетки, содержащиеся в множестве фазового ограничения $\Omega$. Назовем такое множество \textit{возможным}:
\[
    \mathrm{possible}(i,j) = \{
        (i, j) \;|\; i \in \{i, i \pm 1\}, j \in \{j, j \pm 1\}, (x_i,y_j) \in \Omega    
    \}
\]

\textit{Предположение.} На каждом этапе пути можно ехать лишь с некоторой одной скоростью и скорость на предыдущем участке не влияет на время прохода следующего участка.

Данное предположение кажется резонным, если мы рассматриваем, например, автомобиль или человека, а шаг сетки~--- несколько десятков метров: такого расстояния должно хватить для изменения скорости на оптимальную для данного типа поверхности. Для объектов, не удовлетворяющих этому свойству, такое предположение не подходит. Предположение позволяет сократить размернось рассматриваемого фазового пространства с 4 до 2, но превращает управление в импульсное.

Тогда, мы можем считать известным время перехода между точкой $(i,j)$ и любой точкой из ее возможного множества:
\[
    d_{i,j}(\hat i, \hat j),\;\mbox{где $(\hat i, \hat j) \in \mathrm{possible}(i,j)$}.
\]

Таким образом, получили матричное уравнение в неявном виде, которое предстоит решить:
\begin{equation}
    \begin{aligned}
        &V_{i,j} = \min\limits_{(\hat i, \hat j) \in \mathrm{possible}(i, j)}\{d_{i,j}(\hat i,\hat j) + V_{\hat i, \hat j}\}, \\
        &V_{i_1, j_1} = 0.
    \end{aligned}
\end{equation}