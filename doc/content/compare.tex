Ниже приведены времена работы различных версий программы в виде таблиц.
Интересно прослеживается закономерность нелинейного ускорения работы программы для алгоритма Дейкстры в связи с небольшим граничным множеством относительно размера сетки. Так же примерно виден процент времени передачи данных на многонодной установке, относительно времени вычислений.

Приложения запускались на ноутбуке автора с 8 ядерным процессором Intel® Core™ i7-6700HQ и 8 ГиБ ОЗУ.
Многонодная установка состояла из 3 виртуальных машин с 4 ядрами CPU и 8 ГиБ ОЗУ, две из которых использовались для вычислителей. Ноды были взяты у сервиса VK Cloud, который не предоставляет информацию об их физическом расположении и сетевой отдаленности.

\begin{figure}[h]
    \centering
    \begin{tabular}{ | l | l | l | l |}
        \hline
        Размер сетки & Классический & Парал. однонодный & Парал. многонодный \\ \hline
        50$\times$50 & & & \\
        100$\times$100 & 28s & 11s & \\
        250$\times$250 & & & \\
        500$\times$500 & & & \\
        \hline
    \end{tabular}
\caption{Время работы различных реализаций алгоритма Беллмана--Форда.}
\end{figure}

\begin{figure}[h]
    \centering
    \begin{tabular}{ | l | l | l | l |}
        \hline
        Размер сетки & Классический & Парал. однонодный & Парал. многонодный \\ \hline
        500$\times$500 & & & \\
        1000$\times$1000 & 36s & 40s & \\
        2500$\times$2500 & & & \\
        5000$\times$5000 & & & \\
        \hline
    \end{tabular}
    \caption{Время работы различных реализаций алгоритма Дейкстры.}
\end{figure}